\documentclass[twocolumn]{article}
\usepackage{array}
\usepackage[a4paper, total={7in, 10in}]{geometry} % Ajusta los márgenes

\begin{document}

\section*{*}


\subsection*{Configuración de parámetros básicos R2}

\begin{verbatim}
en
conf t
\end{verbatim}

\subsubsection*{Nombre Según la tabla de direccionamiento}

\begin{verbatim}
hostname R2
\end{verbatim}

\subsubsection*{Desactive la búsqueda de DNS}

\begin{verbatim}
no ip domain-lookup
\end{verbatim}

\subsubsection*{Configure la contraseña de enable con su código}

\begin{verbatim}
enable secret 123456
\end{verbatim}

\subsubsection*{Asigne cisco como como la contraseña de consola}

\begin{verbatim}
line vty 0 4
pass cisco
login
\end{verbatim}

\subsubsection*{Mensaje de bienvenida con la palabra Advertencia}

\begin{verbatim}
banner motd *Advertencia*
\end{verbatim}

\subsubsection*{Cifre todas las contraseñas de texto no cifrado}

\begin{verbatim}
service password-encryption
\end{verbatim}

\subsection*{Configuración de acceso por SSH en R2}

\subsubsection*{
    Nombre de dominio cisco.com}

    \begin{verbatim}
    ip domain-name cisco.com
    \end{verbatim}
    
    \subsubsection*{Cree usuario local Admin con contraseña segura P4ssw0rd123}
    
    \begin{verbatim}
    username Admin secret P4ssw0rd123
    \end{verbatim}
    
    \subsubsection*{Genere las claves RSA de 1024 bits}
    
    \begin{verbatim}
    crypto key generate rsa
    1024
    \end{verbatim}
    
    \subsubsection*{Configure las líneas VTY para el acceso por SSH y solicite los perfiles de usuarios locales para la autenticación}
    
    \begin{verbatim}
    line vty 0 4
    transport input ssh
    login local
    exit
    \end{verbatim}
    
    \subsubsection*{Configure una ACL para acceso SSH a los switch y routers donde solo la VLAN de gestión tenga acceso al VTY}
    
    \begin{verbatim}
    access-list 1 permit 192.168.X.128 0.0.0.15
    
    line vty 0 4
    access-class 1 in
    \end{verbatim}
    
    \subsection*{Configuración de opciones de seguridad puertos switch S2}
    
    \subsubsection*{Los puertos de acceso de S2 deben estar en modo acceso}
    
    \begin{verbatim}
    en
    conf t
    hostname S2
    VLAN 10
    name Ventas
    VLAN 11
    name Administrativos
    VLAN 12
    name Servidores
    VLAN 13
    name Gestión
    VLAN 14
    name Native
    
    exit
    int r f 0/1-6
    switchport mode access
    switchport access VLAN 10
    
    int r f 0/7-12
    switchport mode access
    switchport access VLAN 11
    
    int r f 0/13-18
    switchport mode access
    switchport access VLAN 12
    
    int r f 0/19-24
    switchport mode access
    switchport access VLAN 13
    
    int g0/1
    switchport mode trunk
    switchport trunk native VLAN 14
    switchport trunk allowed VLAN 10,11,12,13,14
    \end{verbatim}
    
    \subsubsection*{Deshabilitar los puertos no utilizados}
    
    \begin{verbatim}
    int r f 0/1-24
    shutdown
    int r f0/1
    no sh
    int r f0/7
    no sh
    int r f0/13
    no sh
    \end{verbatim}
    
    \subsubsection*{Habilite la seguridad de puertos en los puertos de los switch}
    
    \begin{verbatim}
    int r f0/1-24
    switchport port-security
    switchport port-security mac-address sticky
    switchport port-security maximum 2
    switchport port-security violation restrict
    \end{verbatim}
    
    \subsection*{Configuración del direccionamiento para todos los dispositivos}
    
    \subsubsection*{Configure router R2 como Server DHCP para las redes LAN1, LAN3, y las VLANs 10 y 11 de R2}
    
    \begin{verbatim}
    ip dhcp excluded-address 192.168.X.1 192.168.X.5
    ip dhcp excluded-address 192.168.X.193 192.168.X.197
    ip dhcp excluded-address 192.168.X.33 192.168.X.37
    ip dhcp excluded-address 192.168.X.65 192.168.X.69
    
    ip dhcp pool LAN-R1
    net 192.168.X.0 255.255.255.224
    default-router 192.168.X.1
    dns-server 192.168.X.98
    exit
    
    ip dhcp pool LAN-R3
    net 192.168.X.192 255.255.255.224
    default-router 192.168.X.193
    dns-server 192.168.X.98
    exit
    
    ip dhcp pool VLAN-10
    net 192.168.X.32 255.255.255.240
    default-router 192.168.X.33
    dns-server 192.168.X.98
    exit
    
    ip dhcp pool VLAN-11
    net 192.168.X.64 255.255.255.240
    default-router 192.168.X.65
    dns-server 192.168.X.98
    exit
    \end{verbatim}
    
    \subsubsection*{En R1}
    
    \begin{verbatim}
    int g0/0
    ip helper-address 192.168.X.33
    \end{verbatim}
    
    \subsubsection*{En R3}
    
    \begin{verbatim}
    int g0/0
    ip helper-address 192.168.X.33
    \end{verbatim}
    
    \subsubsection*{Configure el direccionamiento de manera manual de los demás dispositivos: servidores y gestión de los switch.}
    
    \begin{verbatim}
    en
    conf t
    hostname R3
    int g0/0
    description LAN-R1
    ip add 192.168.X.1 255.255.255.224
    no sh
    int g0/0/0
    description WAN TO R2
    IP ADD 192.168.X.225 255.255.255.252
    no sh
    int g0/1/0
    description WAN TO R3
    ip add 192.168.X.229 255.255.255.252
    no sh
    \end{verbatim}
    
    \subsubsection*{En R2}
    
    \begin{verbatim}
    en
    conf t
    int g0/0
    no sh
    int g0/0.10
    encapsulation dot1Q 10
    ip add 192.168.X.33 255.255.255.240
    
    int g0/0.11
    encapsulation dot1Q 11
    ip add 192.168.X.65 255.255.255.240
    
    int g0/0.12
    encapsulation dot1Q 12
    ip add 192.168.X.97 255.255.255.240
    
    int g0/0.13
    encapsulation dot1Q 13
    ip add 192.168.X.129 255.255.255.240
    
    int g0/0.14
    encapsulation dot1Q 14 native
    ip add 192.168.X.161 255.255.255.240
    
    conf t
    
    int g0/0/0
    description WAN TO R1
    ip add 192.168.X.226 255.255.255.252
    no sh
    
    int g0/1/0
    description WAN TO R3
    ip add 192.168.X.234 255.255.255.252
    no sh
    
    int g0/3/0
    ip add 200.31.12.1 255.255.255.252
    no sh
    description WAN TO ISP
    \end{verbatim}
    
    \subsubsection*{En R3}
    
    \begin{verbatim}
    en
    conf t
    hostname R3
    int g0/0
    description LAN-R3
    IP ADD 192.168.X.193 255.255.255.224
    NO SH
    
    int g0/0/0
    description WAN TO R1
    ip add 192.168.X.230 255.255.255.252
    no sh
    
    int g0/1/0
    description WAN TO R2
    IP ADD 192.168.X.233 255.255.255.252
    NO SH
    \end{verbatim}
    
    \subsection*{Configuración de enrutamiento entre VLANs}
    
    \subsubsection*{Configure las VLANs en S2 de acuerdo a la tabla Vlans. S1 y S3 no se realiza ninguna configuración en este sentido.}
    
    \begin{verbatim}
    en
    conf t
    hostname S2
    VLAN 10
    name Ventas
    VLAN 11
    name Administrativos
    VLAN 12
    name Servidores
    VLAN 13
    name Gestión
    VLAN 14
    name Native
    
    exit
    int r f 0/1-6
    switchport mode access
    switchport access VLAN 10
    
    int r f 0/7-12
    switchport mode access
    switchport access VLAN 11
    
    int r f 0/13-18
    switchport mode access
    switchport access VLAN 12
    
    int r f 0/19-24
    switchport mode access
    switchport access VLAN 13
    
    int g0/1
    switchport mode trunk
    switchport trunk native VLAN 14
    switchport trunk allowed VLAN 10,11,12,13,14
    
    int r f0/1-24
    switchport port-security
    
    switchport port-security mac-address sticky
    switchport port-security maximum 2
    switchport port-security violation restrict
    
    int r f 0/1-24
    shutdown
    
    int f0/1
    no sh
    
    int f0/7
    no sh
    
    int f0/13
    no sh
    
    exit
    int VLAN 13
    no sh
    ip add 192.168.X.130 255.255.255.240
    exit
    ip default-gateway 192.168.X.129
    \end{verbatim}
    
    \subsubsection*{Configure enrutamiento entre VLANs en R2 usando el método de Router on a Stick de acuerdo con la tabla de direccionamiento y la tabla VLANs}
    
    \subsection*{En el server Local de la LAN 2 configure:}
    
    \begin{verbatim}
    SERVER
    
    Configure settings based on the screenshots provided.
    \end{verbatim}
    
    \subsection*{Configuración de enrutamiento dinámico con OSPF}
    
    \subsubsection*{Utilice los siguientes requisitos para configurar el routing OSPF en los tres routers:}
    
    \begin{verbatim}
    router ospf 10
    router-id 1.1.1.1
    net 192.168.X.0 0.0.0.31 area 0
    net 192.168.X.224 0.0.0.3 area 0
    net 192.168.X.228 0.0.0.3 area 0
    passive-interface g0/0
    
    router ospf 10
    router-id 2.2.2.2
    net 192.168.X.0 0.0.0.255 area 0
    passive-interface g0/0
    
    router ospf 10
    router-id 3.3.3.3
    net 192.168.X.192 0.0.0.31 area 0
    net 192.168.X.228 0.0.0.3 area 0
    net 192.168.X.233 0.0.0.3 area 0
    net 192.168.X.232 0.0.0.3 area 0
    passive-interface g0/0
    \end{verbatim}
    
    \subsection*{Configuración de una ruta predeterminada}
    
    \subsubsection*{En R2 configure una ruta predeterminada hacia ISP y garantice que la ruta predeterminada pueda ser compartida por el OSPF a los demás routers}
    
    \begin{verbatim}
    ip route 0.0.0.0 0.0.0.0 g0/3/0
    router ospf 10
    default-information originate
    \end{verbatim}
    
    \subsection*{Configuración de NAT/PAT}
    
    \subsubsection*{Configure NAT estático para el servidor local de acuerdo con la tabla de direccionamiento}
    
    \begin{verbatim}
    ip nat inside source static 192.168.X.98 200.123.226.1
    int g0/0.12
    ip nat inside
    int g0/3/0
    ip nat outside
    \end{verbatim}
    
    \subsubsection*{Configure PAT con overload para todo el rango privado de las 3 redes LAN a través de la IP pública de la WAN entre R2 y el ISP}
    
    \begin{verbatim}
    access-list 2 permit 192.168.X.0 0.0.0.255
    ip nat inside source list 2 interface g0/3/0 overload
    
    int g0/0/0
    ip nat inside
    int g0/1/0
    ip nat inside
    int g0/0.10
    ip nat inside
    int g0/0.11
    ip nat inside
    int g0/0.12
    ip nat inside
    int g0/0.13
    ip nat inside
    int g0/0.14
    ip nat inside
    end
    \end{verbatim}
    
    \subsubsection*{Verifique la conectividad paso a paso y al final y también revise el acceso WEB de los servidores local y externo}
    
    \subsection*{Para completar la tabla}
    
    Verifíquese las interfaces que conectan con las distintas LAN o VLAN, y a cada equipo o switch que pertenezca a dicha red tómese la interfaz que conecta con el Router respectivo. La IP de DNS es la ip privada del servidor DNS (192.168.X.98):


   \begin{tabular}{|>{\centering\arraybackslash}m{3cm}|>{\centering\arraybackslash}m{3cm}|>{\centering\arraybackslash}m{3cm}|}
    \hline
    Dispositivo & Gateway & DNS \\
    \hline
    S1 & 192.168.X.1 & 192.168.X.98 \\
    \hline
    S2 & 192.168.X.129 & 192.168.X.98 \\
    \hline
    S3 & 192.168.X.193 & 192.168.X.98 \\
    \hline
    PC0 & 192.168.X.1 & 192.168.X.98 \\
    \hline
    PC1 & 192.168.X.1 & 192.168.X.98 \\
    \hline
    PC2 & 192.168.X.33 & 192.168.X.98 \\
    \hline
    PC3 & 192.168.X.65 & 192.168.X.98 \\
    \hline
    PC4 & 192.168.X.193 & 192.168.X.98 \\
    \hline
    PC5 & 192.168.X.193 & 192.168.X.98 \\
    \hline
    Server Local & 192.168.X.97 & 192.168.X.98 \\
    \hline
    \end{tabular}   
    
    
    \end{document}